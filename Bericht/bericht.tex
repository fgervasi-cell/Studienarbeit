%&bericht

%%%%%%%%%%%%%%%%%%%%%%%%%%%%%%%%%%%%%%%%%%%%%%%%%%%%%%%%%%%%%%%%%%%%%%%%%%%%%%%
%% Descr:       Vorlage für Berichte der DHBW-Karlsruhe
%% Author:      Prof. Dr. Jürgen Vollmer, juergen.vollmer@dhbw-karlsruhe.de
%% $Id: bericht.tex,v 1.25 2020/03/13 15:07:45 vollmer Exp $
%%  -*- coding: utf-8 -*-
%%%%%%%%%%%%%%%%%%%%%%%%%%%%%%%%%%%%%%%%%%%%%%%%%%%%%%%%%%%%%%%%%%%%%%%%%%%%%%%

\documentclass[
   ngerman          % neue deutsche Rechtschreibung
  ,a4paper          % Papiergrösse
% ,twoside          % Zweiseitiger Druck (rechts/links)
% ,10pt             % Schriftgrösse
  ,11pt
% ,12pt
  ,pdftex
%  ,disable         % Todo-Markierungen auschalten
]{report}

% Bitte die Codierung Ihrer Dateien auswählen:
% \usepackage[latin1]{inputenc}    % Für UNIX mit ISO-LATIN-codierten Dateien
% \usepackage[applemac]{inputenc}  % Für Apple Mac
% \usepackage[ansinew]{inputenc}   % Für Microsoft Windows
\usepackage[utf8]{inputenc}        % UTF-8 codierte Dateien
                                   % Dieses Dokument ist unter Unix erstellt, daher
                                   % wird diese Input-Codierung benutzt.
\setlength{\marginparwidth}{2cm}
\usepackage{bericht}
\usepackage{makecell}

%% ACHTUNG, wenn man eine eigene Formatdatei (bericht.fmt) benutzt, werden Änderungen an bericht.sty
%% erst wirksam, wenn die Format-Datei neu erzeugt wurde!!!
%% Genauer alle Änderungen, die textuell vor der nächsten Zeile ".... endofdump...." stehen
%% werden erst wirksam, wenn die Formatdatei neu erzeugt wurde
\csname endofdump\endcsname

%%listing settings
\lstset{showstringspaces=false}

%%%%%%%%%%%%%%%%%%%%%%%%%%%%%%%%%%%%%%%%%%%%%%%%%%%%%%%%%%%%%%%%%%%%%%%%%%%%%%%
%% Angaben zur Arbeit
%%%%%%%%%%%%%%%%%%%%%%%%%%%%%%%%%%%%%%%%%%%%%%%%%%%%%%%%%%%%%%%%%%%%%%%%%%%%%%%

\newcommand{\Autor}{Felix Manuel Gervasi}
\newcommand{\AutorZwei}{Alena Sutiagina}
\newcommand{\MatrikelNummer}{1052491}
\newcommand{\Kursbezeichnung}{TINF20B4}

\newcommand{\FirmenName}{}
\newcommand{\FirmenStadt}{}
\newcommand{\FirmenLogoDeckblatt}{}

% Falls es kein Firmenlogo gibt:
%  \newcommand{\FirmenLogoDeckblatt}{}

\newcommand{\BetreuerFirma}{Prof. Dr. Jörn Eisenbiegler}

%%%%%%%%%%%%%%%%%%%%%%%%%%%%%%%%%%%%%%%%%%%%%%%%%%%%%%%%%%%%%%%%%%%%%%%%%%%%%%%%%%%%%

% Wird auf dem Deckblatt und in der Erklärung benutzt:
\newcommand{\Was}{Studienarbeit}

%%%%%%%%%%%%%%%%%%%%%%%%%%%%%%%%%%%%%%%%%%%%%%%%%%%%%%%%%%%%%%%%%%%%%%%%%%%%%%%%%%%%%

\newcommand{\Titel}{Bedienung einer Getränkemischmaschine über Sprachbefehle}
\newcommand{\AbgabeDatum}{31. März 2023}

\newcommand{\Dauer}{6 Monate}

\newcommand{\Abschluss}{Bachelor of Science}

\newcommand{\Studiengang}{Angewandte Informatik}

\hypersetup{%%
  pdfauthor={\Autor},
  pdftitle={\Titel},
  pdfsubject={\Was}
}

% um die nervigen Errors zu eliminieren, die auftauchen, wenn man versucht
% einen Zeilenumbruch manuell innerhalb eines section-Titels einzufügen
% s. https://tex.stackexchange.com/questions/10555/hyperref-warning-token-not-allowed-in-a-pdf-string
\pdfstringdefDisableCommands{%
  \def\\{}%
}

%%%%%%%%%%%%%%%%%%%%%%%%%%%%%%%%%%%%%%%%%%%%%%%%%%%%%%%%%%%%%%%%%%%%%%%%%%%%%%%

% Wenn \includeonly{..} benutzt wird, werden nur diese Kaptitel ausgegeben.
%\includeonly{
%  abk
% ,kapitel1
% ,kapitel2
% ,changelog
%}

%%%%%%%%%%%%%%%%%%%%%%%%%%%%%%%%%%%%%%%%%%%%%%%%%%%%%%%%%%%%%%%%%%%%%%%%%%%%%%%

% Benutzt man das "biblatex"-Paket, dann muß das hier stehen:
% siehe auch die mit BIBLATEX markierten Zeilen in bericht.sty
\bibliography{bericht}

\graphicspath{{img/}}

\begin{document}

%%%%%%%%%%%%%%%%%%%%%%%%%%%%%%%%%%%%%%%%%%%%%%%%%%%%%%%%%%%%%%%%%%%%%%%%%%%%%%%

\begin{titlepage}
  \begin{center}
    \vspace*{-2cm}
    \FirmenLogoDeckblatt\hfill\includegraphics[width=4cm]{DHBW_Logo_KA}\\[2cm]
    {\Huge \Titel}\\[1cm]
    {\Huge\scshape \Was}\\[1cm]
    {\large für die Prüfung zum}\\[0.5cm]
    {\Large \Abschluss}\\[0.5cm]
    {\large des Studienganges \Studiengang}\\[0.5cm]
    {\large an der}\\[0.5cm]
    {\large Dualen Hochschule Baden-Württemberg Karlsruhe}\\[0.5cm]
    {\large von}\\[0.5cm]
    {\large\bfseries \Autor}\\[0.5cm]
    {\large und}\\[0.5cm]
    {\large\bfseries \AutorZwei}\\[1cm]
    {\large Abgabedatum \AbgabeDatum}
    \vfill
  \end{center}
  \begin{tabular}{l@{\hspace{2cm}}l}
    Bearbeitungszeitraum          & \Dauer           \\
    Matrikelnummer                & \MatrikelNummer  \\
    Kurs                          & \Kursbezeichnung \\
    Ausbildungsfirma              & \FirmenName      \\
                                  & \FirmenStadt     \\
    Betreuer der Studienarbeit & \BetreuerFirma   \\
  \end{tabular}
\end{titlepage}

%%%%%%%%%%%%%%%%%%%%%%%%%%%%%%%%%%%%%%%%%%%%%%%%%%%%%%%%%%%%%%%%%%%%%%%%%%%%%%%

30 mtime=1584109482.587245501
30 atime=1584109510.325839974
30 ctime=1584109482.587245501


%%%%%%%%%%%%%%%%%%%%%%%%%%%%%%%%%%%%%%%%%%%%%%%%%%%%%%%%%%%%%%%%%%%%%%%%%%%%%%%

\begin{abstract}
  TODO
\end{abstract}

\renewcommand{\abstractname}{Abstract}
\begin{abstract}
  TODO
\end{abstract}

\newpage
\pagenumbering{roman}
\newcounter{roman-numbering}
\tableofcontents           % Inhaltsverzeichnis hier ausgeben
\clearpage
\addcontentsline{toc}{chapter}{Abbildungsverzeichnis}
\listoffigures             % Liste der Abbildungen
\listoftables              % Liste der Tabellen
\clearpage
\addcontentsline{toc}{chapter}{Listingverzeichnis}
\lstlistoflistings         % Liste der Listings
%\listofequations           % Liste der Formeln

% Jetzt kommt der "eigentliche" Text
%%%%%%%%%%%%%%%%%%%%%%%%%%%%%%%%%%%%%%%%%%%%%%%%%%%%%%%%%%%%%%%%%%%%%%%%%%%%%%
%% Descr:       Vorlage für Berichte der DHBW-Karlsruhe, Datei mit Abkürzungen
%% Author:      Prof. Dr. Jürgen Vollmer, vollmer@dhbw-karlsruhe.de
%% $Id: abk.tex,v 1.4 2017/10/06 14:02:03 vollmer Exp $
%% -*- coding: utf-8 -*-
%%%%%%%%%%%%%%%%%%%%%%%%%%%%%%%%%%%%%%%%%%%%%%%%%%%%%%%%%%%%%%%%%%%%%%%%%%%%%%%

\chapter*{Abkürzungsverzeichnis}                   % chapter*{..} -->   keine Nummer, kein "Kapitel"
						         % Nicht ins Inhaltsverzeichnis
\addcontentsline{toc}{chapter}{Akürzungsverzeichnis}   % Damit das doch ins Inhaltsverzeichnis kommt

% Hier werden die Abkürzungen definiert
\begin{acronym}[BIS]
  % \acro{Name}{Darstellung der Abkürzung}{Langform der Abkürzung}
 \acro{Abk}[Abk.]{Abkürzung}

 % Folgendes benutzen, wenn der Plural einer Abk. benöigt wird
 % \newacroplural{Name}{Darstellung der Abkürzung}{Langform der Abkürzung}
 \newacroplural{Abk}[Abk-en]{Abkürzungen}

 \acro{H2O}[\ensuremath{H_2O}]{Di-Hydrogen-Monoxid}

 % Wenn neicht benutzt, erscheint diese Abk. nicht in der Liste
 \acro{NUA}{Not Used Acronym}
 \acro{HTTP}{Hypertext Transfer Protocol}
 \acro{NLP}{Natural Language Processing}
 \acro{AIML}{Artificial Intelligence Markup Language}
 \acro{XML}{eXtensible Markup Language}
 \acro{NLTK}{Natural Language Toolkit}
 \acro{TF}{Term Frequency}
 \acro{IDF}{Inverse Document Frequency}
 \acro{RBM}{Restricted Boltzmann Machines}
 \acro{GAN}{Generative Adversarial Network}
 \acro{RNN}{Recurrent Neural Networks}
 \acro{IoT}{Internet der Dinge}
 \acro{KI}{Künstlichen Intelligenz}
 \acro{ML}{Machine Learning}
 \acro{UI}{User Interface}
 \acro{RAM}{Read Only Memory}
 \acro{CFG}{Context-free grammar}
 \acro{NLTK}{Natural Language Toolkit}
 \acro{API}{Application Programming Interface}
 \acro{CMU}{Carnegie Mellon University}
 \acro{BOW}{Bag Of Words}
 \acro{seq2seq}{sequence-to-sequence}
 \acro{GB}{Gigabyte}
\end{acronym}

              % Abkürzungsverzeichnis
\setcounter{roman-numbering}{\value{page}}
\pagenumbering{arabic}
\chapter{Einleitung}
% Einführung: Interesse vom Leser wecken
Die Informationstechnik versteckt sich heutzutage fast überall - selbst dort, wo sie von den meisten Menschen nicht vermutet werden würde. Beispiele hierfür sind Autos, Kaffeemaschinen, Zahnbürsten, Rasierer, Küchengeräte und vieles mehr. Grund dafür ist die fortschreitende Möglichkeit der Miniaturisierung von Computern, sodass diese nahezu überall verbaut werden können. Beispielsweise können Mikrochips in der Kaffeemaschine dafür sorgen, dass die richtige Menge an Kaffee serviert wird oder der Füllstand der einzelnen Behälter angezeigt werden kann. Solche Systeme, die Informationen mit Hilfe eines Computers verarbeiten und dabei mit ihrer Umgebung derartig \glqq{}verschmelzen\grqq{}, nennt man auch \textit{embedded systems} (z. Dt. \textit{eingebettete Systeme}) \cite{marwedel_eingebettete_2021}.\\\\
Der aktuelle Trend des \ac{IoT} führt zu einem noch größeren Anstieg eingebetteter Systeme im Alltag. Im \ac{IoT} geht es speziell um eingebettete Systeme, die internetfähig (vernetzt) sind. Nach Schätzungen des Marktforschungsunternehmens \textit{Gartner} gab es im Jahr 2017 8,4 Milliarden solcher vernetzten Geräte weltweit \cite{jansen_digitalisierung_2017}. Das die Menge der vernetzten Geräte als Teilmenge der eingebetteten Systeme betrachtet werden kann ist damit zu rechnen, dass deren Anzahl sogar weit größer ausfällt.

\section{Aufgabenstellung}
Im Rahmen dieser Arbeit geht es um die Sprachsteuerung einer Getränkemischmaschine, die in diesem Fall als eingebettetes System zu verstehen ist und in einer vorangegangenen Arbeit bereits konzipiert und gebaut wurde. Sie verfügt derzeit über ein Touch-Display zur Bedienung durch den Benutzer. Ziel der Arbeit ist es zusätzlich eine natürlichsprachliche Interaktion mit der Maschine zu ermöglichen, die mindestens den Funktionsumfang besitzt, der aktuell über die Bildschirmeingabe möglich ist. Dabei soll die Maschine nicht nur in der Lage sein die natrüliche Sprache des Benutzers in ein geeignetes Format umzuwandeln, sodass die Maschine den korrekten Befehl ausführt. Sie soll auch in der Lage sein dem Benutzer zu Antworten, sodass die Illusion einer Konversation mit der Maschine entsteht.

\section{Vorgehen}
Zunächst müssen die Sprachverarbeitung und Spracherkennung betrachtet werden. Die Sprachverarbeitung dient der Interpretation des Gesprochenen um eine geeignete Antwort auszugeben und dem Übersetzen in einen Maschinenbefehl. Im Rahmen dieser Arbeit sollen dafür Verfahren und Techniken der \ac{KI} und des \ac{ML} eingesetzt werden. Die Spracherkennung beschäftigt sich mit der Aufnahme des Tonsignals bzw. der Schallwellen (bspw. über ein Mikrofon) und dem Umwandeln dieser Signale in Text, sodass dieser an das \ac{KI}-Modell weitergereicht werden kann.\\\\
Bei der Arbeit mit eingebetteten Systemen muss man sich der vorhandenen Hardwareleistung und den benötigten Hardwareanforderungen bewusst sein, da diese meist sehr begrenzt ist. Deshalb werden im Rahmen dieser Arbeit verschiedene Ansätze diskutiert, wie und wo die einzelnen Schritte und Berechnungen ablaufen sollen (s. Kapitel \ref{chap:konzept}).
\endinput
\chapter{Stand der Technik}
\section{Getränkemischmaschine}
\section{Hardware}
\section{Sprachverarbeitung}
\subsection{Spracherkennung}
\subsection{Vergleich verschiedener Ansätze}
Derzeit gibt es vier Hauptansätze für die Erstellung eines Chatbots:
\begin{itemize}
    \item Musterabgleich - Musterabgleich und Antwortvorlagen (vorgefertigte Antworten)
    \item Grounding - logische Wissensgraphen und das Ziehen von Schlussfolgerungen aus diesen basierend auf diesen Graphen
    \item Suche - Abrufen von Text
    \item Generierungsmethoden - Statistik und maschinelles Lernen.
\end{itemize}
Die vier grundlegenden Ansätze zur Erstellung von Chatbots lassen sich kombiniert werden, was zu benutzerfreundlichen Chatbots führt. Viele Vielzahl von Anwendungen nutzen alle vier grundlegenden Methoden. Hybride Chatbots unterscheiden sich hauptsächlich darin, wie genau sie diese Ansätze kombinieren und wie viel Gewicht auf jeden einzelnen Ansatz gelegt wird.
\subsubsection{Musterabgleich}
Bei den ersten Chatbots basierte die Antwort auf die Nachricht eines Benutzers auf einem Mustervergleich. Diese Chatbots suchen nach Mustern im eingehenden Text und geben eine feste (gemusterte) Antwort, wenn eine Übereinstimmung gefunden wird \cite{woudenberg_chatbot_2014}.\\\\
Solche rudimentären Dialogsysteme sind vor allem in automatisierten Benutzerunterstützungssystemen mit interaktiven Sprachmenüs nützlich, wo es möglich ist, das Gespräch an einen Menschen weiterzuleiten, wenn der Chatbot keine Antwortmuster mehr hat.\\\\
Da es viele NLP-Dienstprogramme in Python-Paketen gibt, ist es möglich, komplexere Chatbots auf der Grundlage von Mustervergleichen zu erstellen, indem man die Bot-Logik nach und nach direkt in Python mit regulären Ausdrücken und Suchmustern aufbaut.\\\\
1995 machte sich Richard Wallace daran, einen allgemeinen Rahmen für die Erstellung von Chatbots auf der Grundlage des Pattern-Matching-Ansatzes zu schaffen. Zwischen 1995 und 2002 schuf seine Entwicklergemeinschaft die Artificial Intelligence Markup Language (AIML) zur Beschreibung von Mustern und Chatbot-Antworten.\\\\
AIML ist eine deklarative Sprache, die auf dem XML-Standard basiert, der die Sprachkonstrukte und Datenstrukturen einschränkt, die im Bot verwendet werden dürfen. Ein Chatbot, der auf AIML basiert, sieht folgendermaßen aus:
\begin{figure}[H]
    \centering
    \fbox{\includegraphics[width=0.8\textwidth]{Bilder_Kapitel_2/aiml_bot.png}}
    \caption{\label{figure:Aiml_Bot}AIML Chatbot}
\end{figure}
\noindent
Eine der Einschränkungen von AIML ist die Art der Muster, die abgeglichen werden können und auf die reagiert wird. Der AIML-Kern (Pattern Matching Engine) reagiert nur auf Eingabetext, der einem vom Entwickler manuell vorgegebenen Muster entspricht. Unscharfe Suchanfragen, Smileys, Satzzeichen, Tippfehler oder falsch geschriebene Wörter sind nicht erlaubt, es findet kein automatischer Abgleich statt. In AIML müssen alle Synonyme manuell einzeln beschrieben werden.
\subsubsection{Grounding}
Die Grounding-Methode ist ein Ansatz zur Erstellung eines Chatbots auf der Grundlage logischer Wissensgraphen und der Durchführung von Schlussfolgerungen auf der Grundlage dieser Graphen. Sie wird verwendet, um natürliche Sprache zu verarbeiten und sie dem Verständnis des Bots zuzuordnen. Das Wesentliche an der Grounding-Methode ist, dass der Chatbot nicht nur die Textnachrichten, sondern auch den Kontext und die Umgebung verarbeitet, um Anfragen besser zu verstehen und zu beantworten. Durch die Extraktion von Informationen wird ein Netz von Verbindungen oder Fakten geschaffen. Dieses Netz logischer Verbindungen zwischen Entitäten - ein Graph oder eine Wissensbasis - kann die Grundlage für die Antworten des Chatbots bilden.\\\\
Ein Beispiel für eine Grounding-Methode ist die Verwendung eines Wissensgraphen zur Beschreibung der Umgebung. Ein Wissensgraph enthält Informationen über die Objekte, mit denen der Bot interagieren kann, und die Beziehungen zwischen ihnen. Ein Wissensgraph könnte zum Beispiel Informationen über ein Glas auf einem Tisch und das darin befindliche Wasser enthalten. Wenn ein Benutzer eine Frage stellt, verwendet der Chatbot den Wissensgraphen, um den Kontext der Anfrage zu verstehen und die am besten geeignete Antwort abzuleiten. Wenn ein Benutzer zum Beispiel fragt: "Wie hoch ist die Temperatur des Wassers in dem Glas auf dem Tisch?", kann der Chatbot Informationen aus dem Wissensgraphen verwenden, um die Frage zu beantworten.\\\\
Ein solcher Wissensgraph kann abgeleitet werden, um Fragen über die in dieser Wissensbasis enthaltene Welt zu beantworten, und dann können auf der Grundlage der logischen Antworten die Werte der in den Antworten enthaltenen Template-Variablen ausgefüllt werden, um natürlichsprachliche Antworten zu erstellen. Ursprünglich wurden auf diese Weise Systeme zur Beantwortung von Fragen eingerichtet, wie z. B. der Watson-Bot von IBM (heutzutage wird für ähnliche Systeme jedoch die Information Suche Methode verwendet). Der Wissensgraph stellt eine Art 'Erdung' des Chatbots in der realen Welt dar.\\\\
Die Erstellung von Chatbots auf der Grundlage von "Grounding" eignet sich hervorragend für Chatbots, die Fragen generieren, bei denen das zur Beantwortung einer Frage erforderliche Wissen in einer umfangreichen Wissensbasis enthalten ist, die aus einer offenen Datenbank (z. B. Wikidata, Open Mind Common Sense oder DBpedia) bezogen werden kann.\\\\
Einer der Hauptvorteile der Grounding-Methode besteht darin, dass sie sich an ein sich veränderndes Umfeld anpassen kann. Wenn der Benutzer zum Beispiel ein Glas Wasser von einem Tisch auf einen anderen stellt, wird der Wissensgraph automatisch aktualisiert, um diese Änderung widerzuspiegeln.\\\\
Grounding-Methode hat jedoch auch ihre Grenzen. Sie kann bei der Verarbeitung großer Informationsmengen und dadurch eingeschränkt sein, dass sie Zusammenhänge nicht berücksichtigt, die dem Bot möglicherweise verborgen bleiben.\\\\
Insgesamt ist die Grounding-Methode ein effektiver Ansatz zur Erstellung wissensbasierter Chatbots. Sie ermöglicht es dem Bot, Benutzeranfragen besser zu verstehen und eine genauere Antwort zu geben.
\subsubsection{Suche}
Die Informationssuchemethode ist eine der Methoden zum Aufbau von Chatbots, die auf der Extraktion von Informationen aus einer großen Menge von Textinformationen basiert. 
Die Hauptidee der Informationssuchemethode ist die Analyse des Eingabetextes (Benutzeranfrage), die Auswahl von Schlüsselwörtern und Phrasen daraus und die anschließende Suche nach den relevantesten Informationen in der Wissensdatenbank oder in offenen Quellen.\\\\
Die Wissensbasis kann auch eine Art "Gesprächsprotokoll" sein, in Form einer Aussage-Antwort. 
Dabei sucht der Bot nach früheren Aussagen in den Protokollen früherer Unterhaltungen. 
Der Bot kann nicht nur in den Protokollen seiner eigenen Gespräche suchen, sondern auch in beliebigen Transkripten von Gesprächen zwischen Menschen, Gesprächen zwischen Menschen und Bots oder sogar Gesprächen zwischen Bots.
Aber wie immer gilt: Je besser die Eingabedaten, desto besser das Ergebnis. 
Daher ist es notwendig, die Datenbank früherer Gespräche sorgfältig zu säubern und zu organisieren, damit der Bot nach einem qualitativ hochwertigen Gespräch sucht und es dann imitiert.\\\\
Für die Umsetzung der Informationssuchemethode werden verschiedene Algorithmen und Techniken verwendet, z. B. Indizierung und Schlagwortsuche, Kontextsuche, Textanalyse mit Hilfe von maschinellen Lernverfahren usw. 
Die Informationssuchemethode kann in Python mit verschiedenen Bibliotheken und Tools wie NLTK (Natural Language Toolkit), Scikit-learn und Gensim implementiert werden.\\\\
Einer der ersten Schritte bei der Implementierung einer Informationssuchemethode in Python ist die Vorbereitung der Daten. 
Dies erfordert Tokenisierung, Lemmatisierung und die Entfernung von Stopp-Wörtern. 
Als nächstes muss ein Index auf der Grundlage von Schlüsselwörtern erstellt werden. 
Der Index kann auf der Grundlage von Bag-of-Words oder TF-IDF-Modellen (Term Frequency - Inverse Document Frequency) erstellt werden. 
Sobald der Index erstellt ist, kann eine Stichwortsuche durchgeführt werden. 
Dazu muss die Benutzeranfrage in einen Vektor umgewandelt und mit den Dokumentvektoren im Index verglichen werden. 
Dies kann mit Hilfe der Scikit-learn-Bibliothek erfolgen. Sobald die relevantesten Dokumente gefunden wurden, können sie in eine Rangfolge gebracht und als Antwort auf die Benutzeranfrage angezeigt werden.\\\\
Der Vorteil der Informationssuchemethode besteht darin, dass sie ein schnelles und genaues Auffinden der gewünschten Informationen ermöglicht, insbesondere wenn die Wissensbasis gut strukturiert ist und genügend Informationen enthält. 
Ein Nachteil dieser Methode ist jedoch, dass sie den Kontext der Anfrage nicht berücksichtigt und nicht immer eine vollständige und genaue Antwort auf die Frage des Nutzers liefert. 
Wenn die Aussage semantisch mit der vom Bot zu beantwortenden übereinstimmt, ist es möglich, die Antwort wortwörtlich und ohne Änderungen wiederzuverwenden. 
Aber selbst wenn die Datenbank alle möglichen Benutzeräußerungen enthält, wird der Bot die Persönlichkeiten der Personen widerspiegeln, die diese Äußerungen machen. 
Wenn die Antworten konsistent sind und von einer Vielzahl von Personen stammen, ist das gut. 
Problematisch wird es jedoch, wenn die Äußerung, auf die der Bot reagieren soll, vom Gesamtkontext des jeweiligen Gesprächs oder von den Umständen in der Umgebung abhängt, die sich seit der Erstellung des Dialogkorpus geändert haben können.\\\\
Beispielsweise sollte der Bot auf die Frage "Wie spät ist es?" nicht die von der Person gegebene Antwort, sondern die am besten geeignete Aussage aus der Datenbank verwenden. 
Diese Antwort funktioniert nur, wenn die Zeit, zu der die Frage gestellt wurde, mit der Zeit übereinstimmt, zu der die passende Äußerung aus der Datenbank aufgezeichnet wurde. 
Neben dem natürlichsprachlichen Text der Äußerung müssen auch ähnliche Informationen über die Zeit - der Kontext (Zustand) - erfasst und verglichen werden. 
Sie spielt vor allem dann eine wichtige Rolle, wenn die Semantik der Äußerung auf eine aktive Veränderung des im Kontext (Wissensbasis des Chatbots) erfassten Zustands hinweist.\\\\
Um den Zustand (Kontext) in einem Chatbot auf der Grundlage der Informationssuche zu berücksichtigen, kann etwas Ähnliches für einen Chatbot mit Musterabgleich durchgeführt werden, da die Auflistung einer Liste von Benutzeraussagen nur eine andere Art, ein Muster zu beschreiben. 
Dies auch ist der Ansatz von Amazon Lex und Google Dialogflow. 
Anstatt ein starres Muster zu beschreiben, um den Befehl des Benutzers zu erfassen, kann der Dialogflow-Engine einfach ein paar Beispiele geliefert werden. 
So wie jedes Muster im Chatbot auf der Grundlage der Musterzuordnung einem Zustand zugeordnet wurde, muss auch hier nur die Aussage-Antwort-Beispielpaare mit dem genannten Zustand verknüpft werden.\\\\
Der suchbasierte Chatbot indiziert also den Korpus der Dialoge, so dass er leicht frühere Aussagen finden kann, die derjenigen ähnlich sind, auf die er antworten muss, und antwortet dann mit einer der passenden Aussagen aus dem Korpus, die er sich "gemerkt" und für eine schnelle Suche indiziert hat. 
Im Allgemeinen ist die Methode der Informationssuche eine der gängigsten und beliebtesten Methoden zum Aufbau von Chatbots, die in verschiedenen Bereichen wie Wirtschaft, Medizin, Tourismus und vielen anderen eingesetzt werden.\\\\
\subsubsection{Generierungsmethoden}
\endinput
\chapter{Konzept}
In diesem Kapitel wird zunächst erläutert, wie die Steuerung der Getränkemischmaschine durch Sprachbefehle im Allgemeinen ablaufen wird.
Anschließend werden mehrere Konzepte vorgestellt, die das allgemeine Konzept konkretisieren.
Diese werden anhand der, in Kapitel \ref{section:Bewertungskriterien} erläuterten Kriterien, bewertet.
Zuletzt wird die Wahl des finalen Konzepts begründet.
\section{Allgemein}
Der Benutzer soll über Spracheingaben mit der Mischmaschine interagieren können.
Dafür muss das Gesprochene zunächst durch ein Mikrofon aufgenommen werden.
Anschließend können die Audiosignale weiterverarbeitet werden.
Der Benutzer soll hierbei nicht auf fest vorgegebene Sprachbefehle beschränkt sein, sondern für nahezu jede Eingabe eine sinnvolle Antwort zurückerhalten.
Um dies zu gewährleisten wird die Spracheingabe durch ein Sprachmodell, welches mittels maschinellen Lernverfahren trainiert wurde, verarbeitet. 
Ergebnisse dieser Verarbeitung sind die Antwort, die an den Benutzer zurückgegeben wird, und ein konkreter Befehl für die Mischmaschine.
Ein Beispiel für einen solchen Befehl könnte etwa die Zubereitung eines bestimmten Getränks sein.
Für die Ausgabe einer Antwort ist ein Lautsprecher notwendig.
Denkbar wäre auch eine textbasierte Ausgabe, allerdings ginge damit der Eindruck des Benutzers verloren eine echte Konversation mit der Mischmaschine zu führen.
Das Sprachmodell mit der Getränkemischmaschine zu verknüpfen stellt eine Herausforderung dieser Arbeit dar.
\section{Bewertungskriterien} \label{section:Bewertungskriterien}
Im Folgenden sind die Bewertungskriterien für die einzelnen Konzepte aufgelistet:
\begin{itemize}
    \item Freiheitsgrade in der Spracheingabe des Benutzers
    \item Hardwarekosten
    \item Verfügbare Rechenleistung
\end{itemize}
\section{Konzept A: Spracherkennung und -verarbeitung mittels Arduino}
Ein erstes Konzept sieht vor, dass das Audiosignal direkt von einem der Arduinos in der Getränkemischmaschine aufgenommen wird.
\begin{figure}[H]
    \centering
    % \includegraphics does not allow jpg images apparently!
    \fbox{\includegraphics[width=0.8\textwidth]{Bilder_Kapitel_3/Konzept_A.png}}
    \caption{\label{figure:Spracherkennung_mittels_Arduino}Spracherkennung und -verarbeitung mittels Arduino}
\end{figure}
\section{Konzept B: Spracherkennung und -verarbeitung mittels mobiler Anwendung}
\section{Konzept C: Spracherkennung und -verarbeitung auf Computer-Hardware}
\section{Konzept D:}
\section{Finales Konzept}
\endinput




\chapter{Implementierung}
\section{Implementierung des Sprachverarbeitungssystems}
\subsection{Word2Vec-Modell}
\subsection{Sequence-to-Sequence-Modell}
\section{Befehlsverarbeitung in der Mischmaschine}
\section{Anbindung des Sprachmodells an die Mischmaschine}
\endinput



\chapter{Fazit und Ausblick}

TODO





% Ab hier beginnt der Anhang
\pagenumbering{roman}
\setcounter{page}{\value{roman-numbering}}
\appendix
% \addcontentsline{toc}{chapter}{Anhang}

% \addcontentsline{toc}{chapter}{Index}
\printindex

\addcontentsline{toc}{chapter}{Literaturverzeichnis}

% Haben Sie das "biblatex"-Paket nicht installiert, benutzen Sie folgendes:
% Ohne das "biblatex"-Paket (s. bericht.sty) produziert folgendes
% "deutsche" Zitate in Literaturverzeichnissen gemaß der Norm DIN 1505,
% Teil 2 vom Jan. 1984.
% Die Zitatmarken werden alphabetisch nach Verfassern
% sortiert und sind durch abgekürzte Verfasserbuchstaben plus
% Erscheinungsjahr in eckigen Klammern gekennzeichnet.

% \bibliographystyle{alphadin}
% \bibliography{bericht}

%%%%%%%%%%%%%%%%%%%%%%%%%%%%%%%%%%%%%%%5
% BIBLATEX
% Benutzt man das "biblatex"-Paket, muß man folgendes schreiben:
\def\refname{Literaturverzeichnis}
\printbibliography
%%%%%%%%%%%%%%%%%%%%%%%%%%%%%%%%%%%%%%%5


%29 mtime=1584112411.63077173
30 atime=1584112411.634771528
29 ctime=1584112411.63077173


\newpage
\addcontentsline{toc}{chapter}{Liste der ToDo's}
\listoftodos[Liste der ToDo's]


\end{document}
