\chapter{Einleitung}
% Einführung: Interesse vom Leser wecken
Die Informationstechnik versteckt sich heutzutage fast überall - selbst dort, wo sie von den meisten Menschen nicht vermutet werden würde. Beispiele hierfür sind Autos, Kaffeemaschinen, Zahnbürsten, Rasierer, Küchengeräte und vieles mehr. Grund dafür ist die fortschreitende Möglichkeit der Miniaturisierung von Computern, sodass diese nahezu überall verbaut werden können. Beispielsweise können Mikrochips in der Kaffeemaschine dafür sorgen, dass die richtige Menge an Kaffee serviert wird oder der Füllstand der einzelnen Behälter angezeigt werden kann. Solche Systeme, die Informationen mit Hilfe eines Computers verarbeiten und dabei mit ihrer Umgebung derartig \glqq{}verschmelzen\grqq{}, nennt man auch \textit{embedded systems} (z. Dt. \textit{eingebettete Systeme}) \cite{marwedel_eingebettete_2021}.\\\\
Der aktuelle Trend des \ac{IoT} führt zu einem noch größeren Anstieg eingebetteter Systeme im Alltag. Im \ac{IoT} geht es speziell um eingebettete Systeme, die internetfähig (vernetzt) sind. Nach Schätzungen des Marktforschungsunternehmens \textit{Gartner} gab es im Jahr 2017 8,4 Milliarden solcher vernetzten Geräte weltweit \cite{jansen_digitalisierung_2017}. Das die Menge der vernetzten Geräte als Teilmenge der eingebetteten Systeme betrachtet werden kann ist damit zu rechnen, dass deren Anzahl sogar weit größer ausfällt.

\section{Aufgabenstellung}
Im Rahmen dieser Arbeit geht es um die Sprachsteuerung einer Getränkemischmaschine, die in diesem Fall als eingebettetes System zu verstehen ist und in einer vorangegangenen Arbeit bereits konzipiert und gebaut wurde. Sie verfügt derzeit über ein Touch-Display zur Bedienung durch den Benutzer. Ziel der Arbeit ist es zusätzlich eine natürlichsprachliche Interaktion mit der Maschine zu ermöglichen, die mindestens den Funktionsumfang besitzt, der aktuell über die Bildschirmeingabe möglich ist. Dabei soll die Maschine nicht nur in der Lage sein die natrüliche Sprache des Benutzers in ein geeignetes Format umzuwandeln, sodass die Maschine den korrekten Befehl ausführt. Sie soll auch in der Lage sein dem Benutzer zu Antworten, sodass die Illusion einer Konversation mit der Maschine entsteht.

\section{Vorgehen}
Zunächst müssen die Sprachverarbeitung und Spracherkennung betrachtet werden. Die Sprachverarbeitung dient der Interpretation des Gesprochenen um eine geeignete Antwort auszugeben und dem Übersetzen in einen Maschinenbefehl. Im Rahmen dieser Arbeit sollen dafür Verfahren und Techniken der \ac{KI} und des \ac{ML} eingesetzt werden. Die Spracherkennung beschäftigt sich mit der Aufnahme des Tonsignals bzw. der Schallwellen (bspw. über ein Mikrofon) und dem Umwandeln dieser Signale in Text, sodass dieser an das \ac{KI}-Modell weitergereicht werden kann.\\\\
Bei der Arbeit mit eingebetteten Systemen muss man sich der vorhandenen Hardwareleistung und den benötigten Hardwareanforderungen bewusst sein, da diese meist sehr begrenzt ist. Deshalb werden im Rahmen dieser Arbeit verschiedene Ansätze diskutiert, wie und wo die einzelnen Schritte und Berechnungen ablaufen sollen (s. Kapitel \ref{chap:konzept}).

\section{Anforderungen}
Im folgenden sollen die Anforderungen an das Ergebnis der Arbeit konkretisiert werden.
\subsection{Antwortzeit}
Diese Eigenschaft beschreibt die Zeitdauer vom registrieren eines Sprachbefehls bis zur Ausführung des Befehls durch die Mischmaschine und das Zurückgeben einer Antwort an den Benutzer.  Als maximale tolerierbare Antwortzeit werden zehn Sekunden festgelegt. Dies lässt sich mit ... begründen.
\subsection{Offline-Funktionalität}
Die Mischmaschine sollte für die Sprachsteuerung keine Verbindung zum Internet benötigen, da dies die möglichen Einsatzorte der Maschine deutlich einschränken würde.  Diese Anforderung schränkt die möglichen, einzusetzenden Technologien zur Spracherkennung und -verarbeitung stark ein, da keine Cloud-Services wie bspw. \textit{Google Cloud Speech} eingesetzt werden können.
\subsection{Lautstärke}
Die Lautstärke der, von der Mischmaschine zurückgegebenen Antwort, muss laut genug sein, sodass sie vom Benutzer gut verstanden werden kann. Diese Eigenschaft schränkt die Art und Weise wie die Hardware (Computer und Mikrofon) in die Mischmaschine eingebaut werden kann ein und welche Art von Hardware überhaupt verwendet werden kann.
\subsection{Entfernung}
Mit dieser Eigenschaft ist die Entfernung des Anwenders zu der Mischmaschine gemeint. Es muss dem Anwender ermöglicht werden aus einer moderaten Entfernung mit der Mischmaschine über die Sprachsteuerung zu interagieren. Sowohl die Eingabe eines Befehls über die Sprachsteuerung als auch die zu hörende Antwort sollte mindestens aus einer Entfernung von einem Meter möglich sein. Dafür müssen die Lautsprecher eine bestimmte Lautstärke erreichen können und das Mikrofon eine moderate Empfindlichkeit aufweisen.
\subsection{Antworten}
Die Antworten, die durch die Mischmaschine an den Benutzer zurückgegeben werden, sollen mit Hilfe eines eigens erstellten Sprachmodells auf Basis von künstlicher Intelligenz und \ac{ML} erfolgen. Die Antworten der Mischmaschine sollen außerdem bissiger bzw. sarkastischer Natur sein was, je nach verwandter Technik, bei der Auswahl der Trainingsdaten eine große Rolle spielt.  
\subsection{Kosten}
Die Materialkosten sollten einen gewissen Maximalbetrag nicht überschreiten. Zu den benötigten Materialien zählen ein Mikrocomputer zur Durchführung der Spracherkennung und -verarbeitung, ein Mikrofon zur Aufnahme der Sprache und Lautsprecher zur Tonausgabe. Das Ziel besteht darin einen Betrag von ... nicht zu überschreiten.
\subsection{Verbrauch von Arbeits- und Festplattenspeicher}
Diese Anforderung korreliert mit der Anforderung nach Offline-Funktionalität. Diese bedingt, dass aufwendige, rechen- oder speicherintensive Operationen nicht auf entfernte Rechner ausgelagert werden können sondern, alles auf \glqq{}kleiner\grqq{} Hardware innerhalb der Mischmaschine ablaufen muss. Diese Einschränkung soll anhand eines konkreten Beispiels verdeutlicht werden. Die vierte Version der bekannten Mikrocomputerreihe \textit{Raspberry Pi} umfasst im Modell B maximal acht Gigabyte \ac{RAM} und einen Micro-SD-Karten-Steckplatz. Zwar sind Micro-SD-Karten von bis zu mehreren hundert Gigabyte erhältlich, jedoch muss dabei der Kostenfaktor mit beachtet werden. Dadurch sind sowohl Primär- als auch Sekundärspeicher stark beschränkt.\\\\
Auch Prozessorgeschwindigkeit und Grafikkartenleistung können bei der Spracherkennung- und verarbeitung eine Rolle spielen. Diese ist bei Mikrocomputern ebenfalls eingeschränkt und korreliert negativ mit der Anforderung an Geschwindigkeit. Einen ebenso negativen Einfluss auf die Geschwindigkeit des Gesamtsystems haben die virtuelle Vergrößerung des \ac{RAM} durch Swapping oder die SD-Karte selbst, welche im Vergleich mit anderen Speichertechnologien in Sachen Gewschindigkeit deutlich das Nachsehen hat.\\\\
In anbetracht der geschilderten Herausfoderungen und Kostenbetrachtung wird der maximal zu verbrauchende Arbeits- und Festplattenspeicher für dieses Projekt auf ... und ... festgelegt.
\endinput