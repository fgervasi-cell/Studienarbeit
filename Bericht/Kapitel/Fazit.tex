\chapter{Fazit und Ausblick}
Im Zuge dieser Arbeit wurde eine Sprachsteuerung für eine Getränkemischmaschine implementiert, mit deren Hilfe der Benutzer Getränke aus den fünf verfügbaren Behältern mischen kann, indem er die Behälter mit der jeweils gewünschten Prozentangabe nennt. Dafür wurde ein Sprachmodell trainiert, welches anhand eines Textes (Spracheingabe des Benutzers) die Prozentangaben für die jeweiligen Behälter extrahiert und die Art der Eingabe klassifiziert. Der Text kann in eine der drei Klassen eingeordnet werden: Bestellung, Begrüßung, Verabschiedung und keine Eingabe. Je nach Klasse werden bestimmte Antworten ausgewählt, die über Lautsprecher an den Benutzer zurückgegeben werden, damit der Eindruck eines Dialogs zwischen Benutzer und Mischmaschine entsteht. Der Code für die Sprachsteuerung wurde in Python implementiert und wird auf einem Raspberry Pi ausgeführt, der an die bestehende Hardware in der Mischmaschine angeknüpft werden musste.
\section{Erfüllung der Anforderungen}
Am Anfang des Projekts wurden verschiedene Anforderungen an das zu entwickelnde System definiert, die nun einer Überprüfung unterzogen werden sollen.\\\\
Um die Anforderung an die Antwortzeit zu testen wurde die Python-Methode \textit{time()} aus der \textit{time}-Bibliothek verwendet, welche die aktuelle Systemzeit zurückgibt. Dafür wurde in Zeile \ref{code:start_time} des Hauptprogramms aus Anhang \ref{Anhang_B} die Startzeit und in Zeile \ref{code:end_time} die Endzeit gemessen und die Differenz als Antwortzeit berechnet. Der Test wurde zehn mal durchgeführt und es ergab sich hierbei eine mittlere Antwortzeit von 26.66 Sekunden, was deutlich über den gewünschten sechs Sekunden liegt. Besonders Zeitaufwendig sind hierbei die Konvertierung der Audiodaten in Text, die Berechnungen des Sprachmodells bei dem Aufruf von \textit{language\_model.get\_response(text)} als auch die Ausgabe der Antwort über die Lautsprecher, die bei den Tests mit eingeflossen ist und deren Dauer von der ausgewählten Antwort abhängt. Dabei beträgt die Ausführungszeit des Sprachmodells etwas über fünf Sekunden. Als Beispielsatz wurde \glqq{}Mischmaschine. Ich hätte gerne ein Getränk zu 50 Prozent aus Behälter 1 und zu 50 Prozent aus Behälter 2.\grqq{} verwendet. Die Ausführungszeit der Spracherkennung liegt bei etwa 15 Sekunden und macht damit den mit Abstand größten Zeitverlust aus.\\\\
Eine weitere Anforderung war, dass zum Betreiben der Mischmaschine zusammen mit der Sprachsteuerung keine Anbindung zum Internet notwendig sein darf. Diese Anforderung wurde formuliert, damit die Mischmaschine an möglichst vielen Orten betrieben werden kann. Sie wurde in der finalen Umsetzung erreicht, indem alles was für die Sprachsteuerung notwendig ist auf lokaler Hardware ausgeführt wird und auf die Auslagerung von Berechnungen auf Cloud-Services o.ä. verzichtet wurde.\\\\
Lautstärke...\\\\
Entfernung...\\\\
Die Antworten der Mischmaschine sollten humorvoll und kontrollierbar sein und die Kommunikation sollte auf Deutsch stattfinden können. Alle diese Anforderungen wurden im finalen Produkt erreicht. Bei dem gewählten Ansatz für das Sprachmodell werden vordefinierte Antworten je Klasse ausgewählt. Dadurch ist die Grenze bzgl. dem Humor nur durch die eigene Kreativität gesetzt und die Kontrollierbarkeit ist trivialerweise gegeben, da die Antworten eben manuell vorgegeben werden und somit klar ist, was die Mischmaschine in einer bestimmten Situation antworten kann und was nicht. Spracherkennung und -verarbeitung sind auf die deutsche Sprache ausgerichtet.\\\\
Kosten...\\\\
Verbrauch von Arbeits- und Festplattenspeicher...\\\\
Zu guter letzt sollte das System leicht an neue Begebenheiten angepasst werden können. Als Beispiel wurde das Hinzukommen eines neuen Behälters angeführt. In diesem Fall müsste auf Seiten des Arduinos lediglich ein zusätzlicher Prozentwert ausgelesen und damit die Pumpe für den neuen Behälter angesteuert werden. Für das Sprachmodell ergeben sich keine Änderungen (bis auf die Vergrößerung des Arrays für das Speichern der Prozentwerte), da hier bereits eine beliebige Anzahl von Behältern und Prozentwerten extrahiert werden können. Alles in allem ist das Hinzukommen eines neuen Behälters also mit moderatem Aufwand zu bewerkstelligen. Weitere Änderungen die in Zukunft auftreten könnten sind das Ändern der Sprache oder das Hinzufügen neuer Antworten oder Klassen. All dies ist ebenfalls mit moderatem Aufwand zu handhaben. Zum Hinzufügen neuer Antworten reicht das Abändern der \textit{patterns.json}-Datei. Für eine neue Klasse müssten zunächst Muster, also Beispieltexte, für diese Klasse geschrieben und dann das Modell neu trainiert werden. Beim Ändern der Sprache kann für die Spracherkennung eines der Modelle von \textit{OpenAI Whisper} verwendet werden. Die tatsächlichen Änderungen am Code würden sich auf eine einzige Zeile beim Aufruf der \textit{recognize\_whisper}-Methode von der \textit{SpeechRecognition}-Bibliothek beschränken. Das Sprachmodell zur Klassifizierung und Extraktion müsste jedoch neu trainiert werden.
\section{Probleme}
\section{Mögliche Erweiterungen}



