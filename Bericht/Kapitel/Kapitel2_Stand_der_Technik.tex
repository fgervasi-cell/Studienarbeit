\chapter{Stand der Technik}
\section{Getränkemischmaschine}
\section{Hardware}
\section{Sprachverarbeitung}
\subsection{Spracherkennung}
\subsection{Ansätze für die Erstellung eines Chatbots}\label{sec:ansaetze_erstellung_chatbots}
Derzeit gibt es vier Hauptansätze für die Erstellung eines Chatbots(QUELLE! Natural Language Processing in Action: Understanding, analyzing, and generating text with Python
by Hobson Lane, Hannes Hapke, Cole Howard):
\begin{itemize}
    \item Musterabgleich: Musterabgleich und Antwortvorlagen (vorgefertigte Antworten)
    \item Grounding: logische Wissensgraphen und das Ziehen von Schlussfolgerungen aus diesen basierend auf diesen Graphen
    \item Suche: Abrufen von Text
    \item Generierungsmethoden: Statistik und maschinelles Lernen
\end{itemize}
Die vier grundlegenden Ansätze zur Erstellung von Chatbots lassen sich kombinieren, was zu benutzerfreundlicheren Chatbots führt. 
Eine Vielzahl von Anwendungen nutzen alle vier grundlegenden Methoden. 
Hybride Chatbots unterscheiden sich hauptsächlich darin, wie genau sie diese Ansätze kombinieren und wie viel Gewicht auf jeden einzelnen Ansatz gelegt wird.
\subsubsection{Musterabgleich}
Bei den ersten Chatbots basierte die Antwort auf die Nachricht eines Benutzers auf einem Mustervergleich. 
Diese Chatbots suchen nach Mustern im eingehenden Text und geben eine feste (gemusterte) Antwort, wenn eine Übereinstimmung gefunden wird \cite{woudenberg_chatbot_2014}.\\\\
Solche rudimentären Dialogsysteme sind vor allem in automatisierten Benutzerunterstützungssystemen mit interaktiven Sprachmenüs nützlich, wo es möglich ist, das Gespräch an einen Menschen weiterzuleiten, wenn der Chatbot keine Antwortmuster mehr hat.\\\\
Da es viele \ac{NLP}-Dienstprogramme in Python-Paketen gibt, ist es möglich, komplexere Chatbots auf der Grundlage von Mustervergleichen zu erstellen, indem man die Bot-Logik nach und nach direkt in Python mit regulären Ausdrücken und Suchmustern aufbaut.\\\\
1995 machte sich Richard Wallace daran, einen allgemeinen Rahmen für die Erstellung von Chatbots auf der Grundlage des Pattern-Matching-Ansatzes zu schaffen. Zwischen 1995 und 2002 schuf seine Entwicklergemeinschaft die \ac{AIML} zur Beschreibung von Mustern und Chatbot-Antworten.\\\\
\ac{AIML} ist eine deklarative Sprache, die auf dem \ac{XML}-Standard basiert, der die Sprachkonstrukte und Datenstrukturen einschränkt, die im Bot verwendet werden dürfen. (QUELLE http://www.aiml.foundation/doc.html)
Ein Chatbot, der auf \ac{AIML} basiert, sieht folgendermaßen aus:
\begin{figure}[H]
    \centering
    \fbox{\includegraphics[width=0.8\textwidth]{Bilder_Kapitel_2/aiml_bot.png}}
    \caption{\label{figure:Aiml_Bot}\ac{AIML} Chatbot}
\end{figure}
\noindent
Eine der Einschränkungen von \ac{AIML} ist die Art der Muster, die abgeglichen werden können und auf die reagiert wird. 
Der \ac{AIML}-Kern (Pattern Matching Engine) reagiert nur auf Eingabetext, der einem vom Entwickler manuell vorgegebenen Muster entspricht. 
Unscharfe Suchanfragen, Smileys, Satzzeichen, Tippfehler oder falsch geschriebene Wörter sind nicht erlaubt, es findet kein automatischer Abgleich statt. 
In \ac{AIML} müssen alle Synonyme manuell einzeln beschrieben werden.
\subsubsection{Grounding}
Die Grounding-Methode ist ein Ansatz zur Erstellung eines Chatbots auf der Grundlage logischer Wissensgraphen und der Durchführung von Schlussfolgerungen auf der Grundlage dieser Graphen. (QUELLE "Conversational Agents and Natural Language Interaction: Techniques and Effective Practices" Diana Pérez-Marin; Francisco J. Gutierrez-Vela.)
Sie wird verwendet, um natürliche Sprache zu verarbeiten und sie dem Verständnis des Bots zuzuordnen. Das Wesentliche an der Grounding-Methode ist, dass der Chatbot nicht nur die Textnachrichten, sondern auch den Kontext und die Umgebung verarbeitet, um Anfragen besser zu verstehen und zu beantworten. 
Durch die Extraktion von Informationen wird ein Netz von Verbindungen oder Fakten geschaffen. Dieses Netz logischer Verbindungen zwischen Entitäten - ein Graph oder eine Wissensbasis - kann die Grundlage für die Antworten des Chatbots bilden.\\\\
Ein Beispiel für eine Grounding-Methode ist die Verwendung eines Wissensgraphen zur Beschreibung der Umgebung. 
Ein Wissensgraph enthält Informationen über die Objekte, mit denen der Bot interagieren kann und die Beziehungen zwischen ihnen. 
Ein Wissensgraph könnte zum Beispiel Informationen über ein Glas auf einem Tisch und das darin befindliche Wasser enthalten. 
Wenn ein Benutzer eine Frage stellt, verwendet der Chatbot den Wissensgraphen, um den Kontext der Anfrage zu verstehen und die am besten geeignete Antwort abzuleiten. 
Wenn ein Benutzer zum Beispiel fragt: \glqq{}Wie hoch ist die Temperatur des Wassers in dem Glas auf dem Tisch?\grqq{}, kann der Chatbot Informationen aus dem Wissensgraphen verwenden, um die Frage zu beantworten.\\\\
Ein solcher Wissensgraph kann abgeleitet werden, um Fragen über die in dieser Wissensbasis enthaltene Welt zu beantworten und anschließend können auf der Grundlage der logischen Antworten die Werte der in den Antworten enthaltenen Template-Variablen ausgefüllt werden, um natürlichsprachliche Antworten zu erstellen. 
Ursprünglich wurden auf diese Weise Systeme zur Beantwortung von Fragen eingerichtet, wie z. B. der Watson-Bot von IBM (heutzutage wird für ähnliche Systeme jedoch die Informationssuchemethode verwendet). 
Der Wissensgraph stellt eine Art \glqq{}Erdung\grqq{} des Chatbots in der realen Welt dar.\\\\
Die Erstellung von Chatbots auf der Grundlage von \glqq{}Grounding\grqq{} eignet sich hervorragend für Chatbots, die Fragen generieren, bei denen das zur Beantwortung einer Frage erforderliche Wissen in einer umfangreichen Wissensbasis enthalten ist, die aus einer offenen Datenbank (z. B. Wikidata, Open Mind Common Sense oder DBpedia) bezogen werden kann.\\\\
Einer der Hauptvorteile der Grounding-Methode besteht darin, dass sie sich an ein sich veränderndes Umfeld anpassen kann. 
Wenn der Benutzer zum Beispiel ein Glas Wasser von einem Tisch auf einen anderen stellt, wird der Wissensgraph automatisch aktualisiert, um diese Änderung widerzuspiegeln.\\\\
Die Grounding-Methode hat jedoch auch ihre Grenzen. 
So kann es vorkommen, dass bei der Verarbeitung großer Informationsmengen Zusammenhänge nicht berücksichtigt werden und dem Bot möglicherweise verborgen bleiben.\\\\
Insgesamt ist die Grounding-Methode ein effektiver Ansatz zur Erstellung wissensbasierter Chatbots. 
Sie ermöglicht es dem Bot, Benutzeranfragen besser zu verstehen und eine genauere Antwort zu geben.
\subsubsection{Suche}
Die Informationssuchemethode ist eine der Methoden zum Aufbau von Chatbots, die auf der Extraktion von Informationen aus einer großen Menge von Textinformationen basiert. 
Die Hauptidee der Informationssuchemethode ist die Analyse des Eingabetextes (Benutzeranfrage), die Auswahl von Schlüsselwörtern und Phrasen daraus und die anschließende Suche nach den relevantesten Informationen in der Wissensdatenbank oder in offenen Quellen.(QUELLE "Conversational Agents and Natural Language Interaction: Techniques and Effective Practices" Diana Pérez-Marin; Francisco J. Gutierrez-Vela.)\\\\
Die Wissensbasis kann auch eine Art \glqq{}Gesprächsprotokoll\grqq{} sein, in Form von Aussage-Antwort-Paaren. 
Dabei sucht der Bot nach früheren Aussagen in den Protokollen früherer Unterhaltungen. 
Der Bot kann nicht nur in den Protokollen seiner eigenen Gespräche suchen, sondern auch in beliebigen Transkripten von Gesprächen zwischen Menschen, Gesprächen zwischen Menschen und Bots oder sogar Gesprächen zwischen Bots.
Aber wie immer gilt: je besser die Eingabedaten, desto besser das Ergebnis. 
Daher ist es notwendig, die Datenbank früherer Gespräche sorgfältig zu säubern und zu organisieren, damit der Bot nach einem qualitativ hochwertigen Gespräch sucht und es dann imitiert.\\\\
Für die Umsetzung der Informationssuchemethode werden verschiedene Algorithmen und Techniken verwendet, z. B. Indizierung und Schlagwortsuche, Kontextsuche, Textanalyse mit Hilfe von maschinellen Lernverfahren usw. 
Die Informationssuchemethode kann in Python mit verschiedenen Bibliotheken und Tools wie \ac{NLTK}, Scikit-learn und Gensim implementiert werden.\\\\
Einer der ersten Schritte bei der Implementierung einer Informationssuchemethode in Python ist die Vorbereitung der Daten. 
Dies erfordert Tokenisierung, Lemmatisierung und die Entfernung von Stopp-Wörtern. 
Als nächstes muss ein Index auf der Grundlage von Schlüsselwörtern erstellt werden. 
Der Index kann auf der Grundlage von Bag-of-Words oder \ac{TF} und \ac{IDF} (TF-IDF-Modelle) erstellt werden. 
Sobald der Index erstellt ist, kann eine Stichwortsuche durchgeführt werden. 
Dazu muss die Benutzeranfrage in einen Vektor umgewandelt und mit den Dokumentvektoren im Index verglichen werden. 
Dies kann mit Hilfe der Scikit-learn-Bibliothek erfolgen. (QUELLE i WILL SEND YOU A LINK)
Sobald die relevantesten Dokumente gefunden wurden, können sie in eine Rangfolge gebracht und als Antwort auf die Benutzeranfrage angezeigt werden.\\\\
Der Vorteil der Informationssuchemethode besteht darin, dass sie ein schnelles und genaues Auffinden der gewünschten Informationen ermöglicht, insbesondere wenn die Wissensbasis gut strukturiert ist und genügend Informationen enthält. 
Ein Nachteil dieser Methode ist jedoch, dass sie den Kontext der Anfrage nicht berücksichtigt und nicht immer eine vollständige und genaue Antwort auf die Frage des Nutzers liefert. 
Wenn die Aussage semantisch mit der vom Bot zu beantwortenden übereinstimmt, ist es möglich, die Antwort wortwörtlich und ohne Änderungen wiederzuverwenden. 
Aber selbst wenn die Datenbank alle möglichen Benutzeräußerungen enthält, wird der Bot die Persönlichkeiten der Personen widerspiegeln, die diese Äußerungen machen. 
Wenn die Antworten konsistent sind und von einer Vielzahl von Personen stammen, ist das gut. 
Problematisch wird es jedoch, wenn die Äußerung, auf die der Bot reagieren soll, vom Gesamtkontext des jeweiligen Gesprächs oder von den Umständen in der Umgebung abhängt, die sich seit der Erstellung des Dialogkorpus geändert haben können.\\\\
Beispielsweise sollte der Bot auf die Frage \glqq{}Wie spät ist es?\grqq{} nicht die von der Person gegebene Antwort, sondern die am besten geeignete Aussage aus der Datenbank verwenden. 
Diese Antwort funktioniert nur, wenn die Zeit, zu der die Frage gestellt wurde, mit der Zeit übereinstimmt, zu der die passende Äußerung aus der Datenbank aufgezeichnet wurde. 
Neben dem natürlichsprachlichen Text der Äußerung müssen auch ähnliche Informationen über die Zeit - der Kontext (Zustand) - erfasst und verglichen werden. 
Sie spielt vor allem dann eine wichtige Rolle, wenn die Semantik der Äußerung auf eine aktive Veränderung des im Kontext (Wissensbasis des Chatbots) erfassten Zustands hinweist.\\\\
Um den Zustand (Kontext) in einem Chatbot auf der Grundlage der Informationssuche zu berücksichtigen, kann etwas Ähnliches für einen Chatbot mit Musterabgleich durchgeführt werden, da die Auflistung einer Liste von Benutzeraussagen nur eine andere Art ist, ein Muster zu beschreiben. 
Dies auch ist der Ansatz von Amazon Lex (QUELLE Amazon Lex Developer Guide https://docs.aws.amazon.com/lex/latest/dg/how-it-works.html) und Google Dialogflow (Quelle "Building Chatbots with Google Dialogflow: Create chatbots with Dialogflow's natural language processing and machine learning capabilities" by Sumit Chawla). 
Anstatt ein starres Muster zu beschreiben, um den Befehl des Benutzers zu erfassen, können der Dialogflow-Engine einfach ein paar Beispiele geliefert werden. 
So wie jedes Muster im Chatbot auf der Grundlage der Musterzuordnung einem Zustand zugeordnet wurde, muss auch hier nur die Aussage-Antwort-Beispielpaare mit dem genannten Zustand verknüpft werden.\\\\
Der suchbasierte Chatbot indiziert also den Korpus der Dialoge, so dass er leicht frühere Aussagen finden kann, die derjenigen ähnlich sind, auf die er antworten muss und antwortet dann mit einer der passenden Aussagen aus dem Korpus, die er sich \glqq{}gemerkt\grqq{} und für eine schnelle Suche indiziert hat. 
Im Allgemeinen ist die Methode der Informationssuche eine der gängigsten und beliebtesten Methoden zum Aufbau von Chatbots, die in verschiedenen Bereichen wie Wirtschaft, Medizin, Tourismus und vielen anderen eingesetzt werden.
\subsubsection{Generierungsmethoden}
Generierungsmethoden sind einer der wichtigsten Ansätze bei der Entwicklung von Chatbots auf der Grundlage künstlicher Intelligenz. 
Sie ermöglichen es Chatbots, Textantworten auf der Grundlage der Analyse der eingehenden Nachricht und des Kontextes des Dialogs zu generieren. 
Die folgenden Generierungsmodelle sind nützlich, um einen kreativen Chatbot zu erstellen, der Dinge sagen kann, die noch niemand zuvor gesagt hat:
\begin{itemize}
    \item Sequenz-zu-Sequenz-Konvertierungsmodelle: Modelle, die darauf trainiert sind, Antworten auf der Grundlage von Eingabesequenzen zu generieren;
    \item \ac{RBM}: Markov-Ketten, die so trainiert werden, dass sie die \glqq{}Energie\grqq{}-Funktion minimieren (QUELLE "Chatbot Development using Machine Learning Techniques" Nupur Sharma, Anupriya);
    \item \ac{GAN}: statistische Modelle, die darauf trainiert sind, einen Experten, der die Qualität eines Gesprächs bewertet, zu täuschen. (QUELLE "Adversarial Learning for Neural Dialogue Generation" Jiwei Li, Will Monroe, Dan Jurafsky)
\end{itemize}
Die Vorteile des Einsatzes der Generierungsmethoden:
\begin{itemize}
    \item Flexibilität: Generative Methoden können für eine breite Palette von Aufgaben eingesetzt werden, einschließlich Texterstellung, Sprachübersetzung, Verarbeitung natürlicher Sprache und mehr.
    \item Automatisierung: Generative Methoden können auf großen Datensätzen trainiert werden, wodurch die Erstellung von Inhalten automatisiert werden kann.
    \item Qualität: Generative Methoden zeigen eine hohe Qualität bei der Textgenerierung, Sprachübersetzung und anderen Aufgaben der natürlichen Sprachverarbeitung, wenn sie auf einem ausreichend großen Datensatz trainiert werden.
    \item Schnelligkeit: Generative Methoden können schneller arbeiten als Menschen, was die Erstellung von Inhalten mit großer Geschwindigkeit ermöglicht.
\end{itemize}
Die Nachteile der generativen Methoden:
\begin{itemize}
    \item Große Datenmengen für das Training: Generative Methoden benötigen große Datenmengen für das Training, was bei einigen Aufgaben schwierig sein kann, insbesondere wenn nur ein kleiner Datensatz zur Verfügung steht.
    \item Sicherheitsrisiken: Generative Methoden können Inhalte erzeugen, die möglicherweise falsch, unvollständig oder irreführend sind. Dies kann zu Sicherheitsrisiken führen, wenn der generierte Inhalt für wichtige Entscheidungen verwendet wird.
    \item Unterstützungsbedarf: Generative Methoden können erhebliche Unterstützung benötigen, um effektiv zu sein. Dies kann die Modellabstimmung, die Auswahl optimaler Parameter und die Optimierung der Modellleistung auf einer bestimmten Hardwarekonfiguration umfassen.
    \item Modellbeschränkungen: Generative Methoden können Beschränkungen hinsichtlich der Arten von Inhalten haben, die sie erzeugen können, insbesondere wenn sie nur auf bestimmte Datentypen trainiert wurden.
\end{itemize}
Eine der beliebtesten Methoden zur Texterstellung ist die sequence-to-sequence-Methode (seq2seq). 
Die seq2seq-Methode basiert auf \ac{RNN}, die die Simulation von Datenfolgen ermöglichen. 
Sie besteht aus zwei Hauptteilen: einem Encoder und einem Decoder. 
Ein Encoder empfängt eine Wortfolge und baut daraus einen Kontextvektor auf, der Informationen über die Eingabedaten enthält. 
Der Decoder erhält diesen Vektor als Eingabe und beginnt mit der Generierung einer Folge von Antwortnachrichten, wobei er schrittweise den Kontext und die zuvor generierten Wörter berücksichtigt.\\\\
Einer der Hauptvorteile der seq2seq-Methode ist ihre Fähigkeit, qualitative und grammatikalisch korrekte Textantworten zu generieren, einschließlich Antworten, die nicht in den Trainingsdaten enthalten waren. 
Sie kann auch mit langen Sequenzen umgehen, was sie ideal für die Generierung von Antworten in Dialogsystemen macht. 
Darüber hinaus kann die seq2seq-Methode in einer Vielzahl von Anwendungen eingesetzt werden, z. B. in der maschinellen Übersetzung, der Spracherkennung und anderen.\\\\
Die seq2seq-Methode hat jedoch ihre Nachteile. 
Sie erfordert große Datenmengen zum Trainieren und Verarbeiten sowie erhebliche Rechenressourcen. 
Dies kann die Anwendung der Methode bei einigen Anwendungen einschränken. 
Wenn der Trainingsdatensatz nicht eine ausreichend große Bandbreite möglicher Antworten repräsentiert, kann das Modell außerdem dazu neigen, vorhersehbare oder falsche Antworten zu erzeugen.\\\\
Die Implementierung der seq2seq-Methode in Python kann mit der TensorFlow-Bibliothek erfolgen, die eine Reihe von Werkzeugen für den Aufbau und das Training neuronaler Netze bietet. 
In TensorFlow kann man die vortrainierten seq2seq-Modelle verwenden oder ein eigenes Modell erstellen, indem die Architektur und die Trainingsparameter des Netzwerks konfiguriert wird.
\endinput