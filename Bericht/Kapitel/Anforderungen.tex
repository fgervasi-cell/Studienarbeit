\chapter{Anforderungen} \label{chap:Anforderungen}
Im Folgenden sollen die Anforderungen an das Ergebnis der Arbeit konkretisiert werden.
\section{Antwortzeit}
Diese Eigenschaft beschreibt die Zeitdauer vom registrieren eines Sprachbefehls bis zur Ausführung des Befehls durch die Mischmaschine und das Zurückgeben einer Antwort an den Benutzer. Die Antwortzeit spielt eine große Rolle bei der Bedienbarkeit eines interaktiven Systems, um das es sich bei der Mischmaschine handelt. Zu lange Antwortzeiten können dazu führen, dass der Benutzer seine ursprünglichen Ziele vergisst oder in Stress gerät, da in den aller meisten Fällen der Grund für eine lange Antwortzeit vor dem Benutzer verborgen bleibt. Umgekehrt können zu kurze Antwortzeiten ebenfalls zu Stress und Fehlbedienung seitens des Benutzers führen. Dies liegt unter anderem daran, dass kurze Antwortzeiten den Benutzer dazu veranlassen weniger über seine Aktionen und deren Folgen nachzudenken. Als eine für viele Anwendungen geeignete Antwortzeit werden zwei bis vier Sekunden genannt \cite{herczeg_9_2018}. Als maximale tolerierbare Antwortzeit werden für dieses Projekt sechs Sekunden festgelegt. Diese vergleichsweise lange Zeitdauer lässt sich zum Einen mit den langwierigen aber notwendigen Berechnungen begründen, die für die Spracherkennung und -verarbeitung benötigt werden. Zum Anderen werden die Auswirkungen einer langen Antwortzeit als gering eingeschätzt, da der Benutzer für diese Anwendung keine Teilarbeitsschritte o. ä. im Gedächtnis behalten muss. Das Ziel des Benutzers sich ein Getränk zubereiten zu lassen ist nach dem Eingang des Sprachbefehls bereits erfüllt.
\section{Offline-Funktionalität}
Die Mischmaschine sollte für die Sprachsteuerung keine Verbindung zum Internet benötigen, da dies die möglichen Einsatzorte der Maschine deutlich einschränken würde.  Diese Anforderung schränkt die möglichen, einzusetzenden Technologien zur Spracherkennung und -verarbeitung stark ein, da keine Cloud-Services wie bspw. \textit{Google Cloud Speech} eingesetzt werden können \cite{google_cloud_speech}. Eine weitere Herausforderung die dadurch entsteht ist, dass Berechnungen die unter Umständen sehr aufwendig sein können nicht ausgelagert sondern auf der Hardware innerhalb der Mischmaschine ausgeführt werden müssen.
\section{Lautstärke}
Die Lautstärke der, von der Mischmaschine zurückgegebenen Antwort, muss laut genug sein, sodass sie vom Benutzer gut verstanden werden kann. Diese Eigenschaft schränkt die Art und Weise wie die Hardware (Computer und Mikrofon) in die Mischmaschine eingebaut werden kann ein und welche Art von Hardware überhaupt verwendet werden kann.
\section{Entfernung}
Mit dieser Eigenschaft ist die Entfernung des Anwenders zu der Mischmaschine gemeint. Es muss dem Anwender ermöglicht werden aus einer moderaten Entfernung mit der Mischmaschine über die Sprachsteuerung zu interagieren. Sowohl die Eingabe eines Befehls über die Sprachsteuerung als auch die zu hörende Antwort sollte mindestens aus einer Entfernung von einem Meter möglich sein. Dafür müssen die Lautsprecher eine bestimmte Lautstärke erreichen können und das Mikrofon eine moderate Empfindlichkeit aufweisen.
\section{Antworten}
Die Antworten, die durch die Mischmaschine an den Benutzer zurückgegeben werden, sollen mit Hilfe eines eigens erstellten Sprachmodells auf Basis von künstlicher Intelligenz und \ac{ML} erfolgen. Die Antworten der Mischmaschine sollen außerdem bissiger bzw. sarkastischer Natur sein was, je nach verwandter Technik, bei der Auswahl der Trainingsdaten eine große Rolle spielt.\\\\
Des Weiteren bestehen die Anforderungen, dass der Benutzer auf Deutsch mit der Mischmaschine kommunizieren können muss und die Antworten der Mischmaschine kontrollierbar sein müssen. Mit der Kontrollierbarkeit ist gemeint, dass Vorhersagen darüber gemacht werden können, was die Mischmaschine in Etwa auf eine bestimmte Frage oder sonstige Benutzereingabe antworten wird. Dies soll verhindern, dass der Benutzer von unerwarteten Reaktionen seitens der Maschine überrascht wird und die Antwroten der Maschine den Benutzer nicht beleidigen (aufgrund der sarkastischen bzw. humorvollen Art und Weise, wie die Maschine antworten soll).     
\section{Kosten}
Die Materialkosten sollten einen gewissen Maximalbetrag nicht überschreiten. Zu den benötigten Materialien zählen ein Mikrocomputer zur Durchführung der Spracherkennung und -verarbeitung, ein Mikrofon zur Aufnahme der Sprache und Lautsprecher zur Tonausgabe. Das Ziel besteht darin einen Betrag von 200€ nicht zu überschreiten.
\section{Verbrauch von Arbeits- und Festplattenspeicher}
Diese Anforderung korreliert mit der Anforderung nach Offline-Funktionalität. Diese bedingt, dass aufwendige, rechen- oder speicherintensive Operationen nicht auf entfernte Rechner ausgelagert werden können sondern, alles auf \glqq{}kleiner\grqq{} Hardware innerhalb der Mischmaschine ablaufen muss. Diese Einschränkung soll anhand eines konkreten Beispiels verdeutlicht werden. Die vierte Version der bekannten Mikrocomputerreihe \textit{Raspberry Pi} umfasst im Modell B maximal acht Gigabyte \ac{RAM} und einen Micro-SD-Karten-Steckplatz \cite{ltd_raspberry_nodate}. Zwar sind Micro-SD-Karten von bis zu mehreren hundert Gigabyte erhältlich, jedoch muss dabei der Kostenfaktor mit beachtet werden. Dadurch sind sowohl Primär- als auch Sekundärspeicher stark beschränkt.\\\\
Auch Prozessorgeschwindigkeit und Grafikkartenleistung können bei der Spracherkennung- und verarbeitung eine Rolle spielen. Diese ist bei Mikrocomputern ebenfalls eingeschränkt und korreliert negativ mit der Anforderung an Geschwindigkeit. Einen ebenso negativen Einfluss auf die Geschwindigkeit des Gesamtsystems haben die virtuelle Vergrößerung des \ac{RAM} durch Swapping oder die SD-Karte selbst, welche im Vergleich mit anderen Speichertechnologien in Sachen Geschwindigkeit deutlich das Nachsehen hat.\\\\
In anbetracht der geschilderten Herausfoderungen und Kostenbetrachtung wird der maximal zu verbrauchende Arbeits- und Festplattenspeicher für dieses Projekt auf acht Gigabyte \ac{RAM} und 20 Gigabyte Festplattenspeicher festgelegt.
\section{Anpassungsfähigkeit}
Diese Anforderung beschreibt den Grad der Einfachheit bei Änderung der Anforderungen oder Umwelt die Sprachsteuerung der Mischmaschine an diese neuen Begebenheiten anzupassen. Ein einfaches Beispiel für die Anpassungsfähigkeit des Systems wäre das Hinzukommen eines Behälters innerhalb der Mischmaschine. Tritt dieser Fall ein sollte es leicht möglich sein die Sprachsteuerung so anzupassen, dass der Benutzer auch die Möglichkeit bekommt aus dem neuen, fünften Behälter Getränke zu Mischen und zu Bestellen. 
\section{Bewertung der Ansätze zur Erstellung eines Dialogsystems}
Unter Berücksichtigung der genannten Anforderungen und der im Kapitel \ref{sec:ansaetze_erstellung_chatbots} besprochenen Ansätze zur Erstellung eines Dyalogsystems, können die folgenden Schlussfolgerungen gezogen werden, ob die Ansätze den Anforderungen entsprechen oder nicht:
\begin{table}[H]
    \centering
    \begin{tabular}{l|c|c|c|c}
        \makecell{Ansatz/\\Anforderung} & Musterabgleich & Grounding & Suche & \makecell[l]{Generierungs-\\methoden} \\
        \hline
        \makecell{Antwortzeit} & entspricht  & entspricht nicht & entspricht & entspricht nicht \\
        \hline
        \makecell{Offline-\\Funktionalität}            & entspricht             & entspricht      & entspricht   & entspricht \\
        \hline
        \makecell{Sarkastische \\Antworten} & entspricht      & entspricht & entspricht & entspricht nicht   \\
        \hline
        \makecell{Antworten \\auf deutsch}                & entspricht     & entspricht nicht    & entspricht nicht    & entspricht nicht \\
        \hline
        \makecell{kontrollierbare \\Antworten}                  & entspricht             & entspricht      & entspricht & entspricht nicht   \\
        \hline
        \makecell{Verbrauch von \\Arbeits- und \\Festplattenspeicher}             & entspricht             & entspricht nicht      & entspricht nicht & entspricht nicht   \\
    \end{tabular}
    \caption{\label{table:Bewertungsmatrix_Anforderungen_Dialogsystem}Bewertung der Ansätze zur Erstellung eines Dialodsystems}
\end{table}
\noindent
\endinput